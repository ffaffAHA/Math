%空头
% UTF8编码,ctexart现实中文
%删除了\usepackage{color}
% 使用颜色
\definecolor{orange}{RGB}{255,127,0} 
\definecolor{violet}{RGB}{192,0,255} 
\definecolor{aqua}{RGB}{0,255,255}
%删除了\usepackage{geometry}
\setcounter{tocdepth}{4}
\setcounter{secnumdepth}{4}
% 设置四级目录与标题
%删除A4
% 默认大小为A4
%删除页面边距
% 默认页边距为1英尺与1.25英尺
%删除了\usepackage{indentfirst}
%删除首行缩进
% 首行缩进2个中文字符
%删除\usepackage{setspace}
\renewcommand{\baselinestretch}{1.5}
% 1.5倍行距
%删除\usepackage{amssymb}
% 因为所以
%删除\usepackage{amsmath}
% 数学公式
%\usepackage[colorlinks,linkcolor=black,urlcolor=blue]{hyperref}
% 超链接
%删除了作者
\chapter{微分方程}
%删除了页面格式

本节内容较少。

若一曲线过点$(1,2)$,且该曲线上任一点$M\,(x,y)$处的切线的斜率为$2x$,求该曲线的方程。

令所求曲线为$\varphi(x)$,$\dfrac{\textrm{d}y}{\textrm{d}x}=2x$,且$x=1$时,$y=2$。

两边积分:$\int\textrm{d}y=y=\int2x\,\textrm{d}x$。所以$y=x^2+C$。

代入$(1,2)$,$C=1$,所以$y=x^2+1$。

\section{微分方程基本概念}

\subsection{微分方程构成}

\textcolor{violet}{\textbf{定义:}}表示未知函数、未知函数的导数与自变量之间的关系的方程,即含导数的方程就是\textbf{微分方程}。导数可能是一阶导数也可能是二阶以及以上阶数的导数。

常微分方程\textcolor{violet}{\textbf{定义:}}未知函数是一元函数的微分方程。如$y'''-y''+6y=0$,$y\,\textrm{d}x-(x+\sqrt{x^2+y^2})\,\textrm{d}y=0$。

\textcolor{violet}{\textbf{定义:}}微分方程所出现的未知函数的最高阶导数的阶数就是该微分方程的\textbf{阶}。

$n$阶微分方程的形式是$F(x,y,y',\cdots,y^{(n)})=0$。其中最高阶导数是必须出现的。若能从中解出最高阶导数,则可得微分方程$y^{(n)}=f(x,y,y',\cdots,y^{(n-1)})$。

\subsection{微分方程的解}

微分方程的解是函数。

\textcolor{violet}{\textbf{定义:}}若微分方程中的解中含有任意常数,且任意常数的个数与微分方程的阶数相同,则就是微分方程的\textbf{通解}。

如若$y''=3$,则$y'=3x+C_1$,$y=\dfrac{3}{2}x^2+C_1x+C_2$,此时含有两个任意常数$C_1C_2$,则微分方程的阶数也为2。

\textcolor{violet}{\textbf{定义:}}确定通解中任意常数后,就得到微分方程的\textbf{特解}。

\textcolor{violet}{\textbf{定义:}}当给出$x=x_0$时$y_0$与$y_0'$的值,那么这些条件就是\textbf{初值条件},如上面的$y''=3$。

求微分方程$y'=f(x,y)$满足初值条件$y\vert_{x=x_0}=y_0$的特解这样的问题,就是一阶微分方程的初值问题,记为$\left\{\begin{array}{l}
    y'=f(x,y) \\
    y\vert_{x=x_0}=y_0
\end{array}
\right.$。

微分方程的解的图形是一条曲线,叫做微分方程的\textbf{积分曲线},初值问题的集几何意义就是求微分方程的通过某点的积分曲线。

\textbf{例题:}判断函数$x=C_1\cos kt+C_2\sin kt$是否是微分方程$\dfrac{\textrm{d}^2x}{\textrm{d}t^2}+k^2x=0$的解,若是则令其为$k\neq0$时方程的通解,求满足初值条件$x\vert_{t=0}=A$,$\dfrac{\textrm{d}x}{\textrm{d}t}\bigg\vert_{t=0}=0$时的特解。

解:判断是否为方程的解,就要将这个解代入微分方程中。微分方程中除了$x$,还出现了$x''$,所以需要先将$x$对$t$求两次导:

$x'=-kC_1\sin kt+kC_2\cos kt$,$x''=-k^2C_1\sin kt-k^2C_2\sin kt$。代入方程:

$-k^2(C_1\sin kt+C_2\sin kt)+k^2(C_1\cos kt+C_2\sin kt)\equiv0$,所以是解,然后求特解:

代入$x\vert_{t=0}=A$,$\therefore C_1=A$,代入$\dfrac{\textrm{d}x}{\textrm{d}t}\bigg\vert_{t=0}=0$,$\therefore C_2=0$。

所以代入$x=C_1\cos kt+C_2\sin kt$得到特解:$x=A\cos kt$。

\section{可分离变量的微分方程}

对于第一节的$\textrm{d}y=2x\,\textrm{d}x$可以直接求解,如$\dfrac{\textrm{d}y}{\textrm{d}x}=2x$直接移项就可以得到通解$x^2+C$。

但是并不是所有都是如此,如$\dfrac{\textrm{d}y}{\textrm{d}x}=2xy^2$求积分得$y=\int2xy^2\,\textrm{d}x$,这本身不能直接解,但是可以将$\dfrac{\textrm{d}y}{\textrm{d}x}=2xy^2$先两边同乘$\dfrac{\textrm{d}x}{y^2}$得到$\dfrac{\textrm{d}y}{y^2}=2x\textrm{d}x$,将$xy$分离在两端,然后两边同时积分得到$-\dfrac{1}{y}=x^2+C$,所以$y=-\dfrac{1}{x^2+C}$。

\textcolor{violet}{\textbf{定义:}}形如$y'=f(x)g(y)$的方程就是\textbf{变量可分离型}方程。

可以变型为$\dfrac{\textrm{d}y}{g(y)}=f(x)\textrm{d}x$,即将含$y$的放在一边,含$x$的放在另一边。然后对两边求积分就得到$\displaystyle{\int\dfrac{\textrm{d}y}{g(y)}=\int f(x)\textrm{d}x}$,解得隐式解或隐式通解$G(y)=F(x)+C$。最后可以将隐式解化为显式解。

\textbf{例题:}求微分方程$\dfrac{\textrm{d}y}{\textrm{d}x}=2xy$。

解:$\displaystyle{\int\dfrac{\textrm{d}y}{y}}=\int2x\,\textrm{d}x$,$\ln\vert y\vert=x^2+C$,$\vert y\vert=e^{x^2+C}$。

$\therefore y=\pm e^{x^2}e^C=\pm C_1e^{x^2}=C_2e^{x^2}$。

\textcolor{orange}{注意:}在微分方程部分可以直接$\ln y=x^2+C$而不用管正负号,其$C$的正负号由取指后左边式子决定,如果左边为正数如$\sqrt{u}$的形式则$C$也为正数,而这个式子左边为$y$,所以为任意常数。

\section{可化为可分离变量型}

\subsection{多项式换元}

形如$\dfrac{\textrm{d}y}{\textrm{d}x}=f(ax+by+c)$的方程,其中$a,b,c$全不为0。

令$u=ax+by+c$,则$\dfrac{\textrm{d}u}{\text{d}x}=a+b\dfrac{\textrm{d}y}{\textrm{d}x}$,代入原方程$\dfrac{\textrm{d}u}{\textrm{d}x}=a+bf(u)$。

\subsection{自然齐次方程}

若一阶微分方程可化为$\dfrac{\textrm{d}y}{\textrm{d}x}=\varphi\left(\dfrac{y}{x}\right)$,则这方程就是一个齐次方程。

也可能出现$\dfrac{\textrm{d}x}{\textrm{d}y}=\varphi\left(\dfrac{x}{y}\right)$。

令$u=\dfrac{y}{x}$,则$y=ux$变为$\dfrac{\textrm{d}y}{\textrm{d}x}=u+x\dfrac{\textrm{d}u}{\textrm{d}x}$($u$不是一个常数而是一个关于$x$的函数,所以$\dfrac{\textrm{d}y}{\textrm{d}x}\neq u$),从而原方程变为$x\dfrac{\textrm{d}u}{\textrm{d}x}+u=\varphi(u)$,即$\dfrac{\textrm{d}u}{\varphi(u)-u}=\dfrac{\textrm{d}x}{x}$。

如$(xy-y^2)\textrm{d}x-(x^2-2xy)\textrm{d}y=0$可以化为$\dfrac{\textrm{d}y}{\textrm{d}x}=\dfrac{xy-y^2}{x^2-2xy}$,即$\dfrac{\textrm{d}y}{\textrm{d}x}=\dfrac{\dfrac{y}{x}-\left(\dfrac{y}{x}\right)^2}{1-2\left(\dfrac{y}{x}\right)}$。

解决齐次方程问题的过程:令$u=\dfrac{y}{x}$;$y=xu$;$\dfrac{\textrm{d}y}{\textrm{d}x}=u+x\dfrac{\textrm{d}u}{\textrm{d}x}$。

代入微分方程:$u+x\dfrac{\textrm{d}u}{\textrm{d}x}=\varphi(u)$,$\therefore x\dfrac{\textrm{d}u}{\textrm{d}x}=\varphi(u)-u$,分离变量:$\dfrac{\textrm{d}u}{\varphi(u)-u}=\dfrac{\textrm{d}x}{x}$,求积分$\displaystyle{\int\dfrac{\textrm{d}u}{\varphi(u)-u}=\int\dfrac{\textrm{d}x}{x}}$。最后求出积分再用$\dfrac{y}{x}$替代$u$。

若是方程可以变为齐次方程,则$x$和$y$的幂应该是对称的,可以尝试除以一个$x^a$来变为$\dfrac{y^a}{x^a}$形式。

\textbf{例题:}求$y^2+x^2\dfrac{\textrm{d}y}{\textrm{d}x}=xy\dfrac{\textrm{d}y}{\textrm{d}x}$。

解:得到$\dfrac{\textrm{d}y}{\textrm{d}x}=\dfrac{y^2}{xy-x^2}$。

然后将这个等式化为$\dfrac{y}{x}$的形式,分子分母同时除以$x^2$:$\dfrac{\dfrac{y^2}{x^2}}{\dfrac{xy-x^2}{x^2}}=\dfrac{\left(\dfrac{y}{x}\right)^2}{\dfrac{y}{x}-1}$。

从而到第三步:$u+x\dfrac{\textrm{d}u}{\textrm{d}x}=\dfrac{u^2}{u-1}$,$\therefore x\dfrac{\textrm{d}u}{\textrm{d}x}=\dfrac{u^2}{u-1}-u=\dfrac{u}{u-1}$。

$\therefore\dfrac{u-1}{u}\textrm{d}u=\dfrac{\textrm{d}x}{x}$,$\therefore\displaystyle{\int\dfrac{u-1}{u}\textrm{d}u=\int\dfrac{\textrm{d}x}{x}}$,$u-\ln u=\ln x+C$,$\ln xu=u+C$。

代入$u=\dfrac{y}{x}$,得到$\ln y=\dfrac{y}{x}+C$,所以得到$y=Ce^{\frac{y}{x}}$。

\subsection{可化为齐次方程}

对于自然齐次方程,其形式如$\dfrac{\textrm{d}y}{\textrm{d}x}=\dfrac{A_1x+B_1y}{A_2x+B_2y}$,则可以除以$x$得到齐次方程。

而对于形式如$\dfrac{\textrm{d}y}{\textrm{d}x}=\dfrac{A_1x+B_1y+C_1}{A_2x+B_2y+C_2}$,则因为有常数项,所以不能直接除以$x$。

所以想尝试消去常数项。令$x=X+h$,$y=Y+k$。\medskip

$\therefore\dfrac{\textrm{d}Y}{\textrm{d}X}=\dfrac{A_1X+B_1Y+A_1h+B_1k+C_1}{A_2X+B_2Y+A_2h+B_2k+C_2}$,当取一个合适的$h$和$k$时常数项$A_1h+B_1k+C_1=A_2h+B_2k+C_2=0$,从而能化为齐次方程。

若$\dfrac{A_2}{A_1}\neq\dfrac{B_2}{B_1}$,则可以解得:\medskip

$\left\{\begin{array}{l}
    k=\dfrac{A_1C_2-A_2C_1}{A_2B_1-A_1B_2} \\  
    h=\dfrac{A_1B_1C_2-A_2B_1C_1+A_1A_2B_1C_1-A_1^2B_2C_1}{A_1^2B_2-A_1A_2B_1}
\end{array}
\right.$.

若$\dfrac{A_2}{A_1}=\dfrac{B_2}{B_1}$,即关系式对应成比例。

令$\dfrac{A_2}{A_1}=\dfrac{B_2}{B_1}=\lambda$,$\therefore\dfrac{\textrm{d}y}{\textrm{d}x}=\dfrac{A_1x+B_1y+C_1}{\lambda(A_1x+B_1y)+C_2}$。

又令$A_1x+B_1y=v$,$\therefore\dfrac{\textrm{d}v}{\textrm{d}x}=A_1+B_1\,\dfrac{\textrm{d}y}{\textrm{d}x}$,$\dfrac{\textrm{d}v}{\textrm{d}x}=A_1+B_1\dfrac{v+C_1}{\lambda v+C_2}$

$=\dfrac{(A_1\lambda+B_1)v+A_1C_2+B_1C_1}{\lambda v+C_2}$。此时未知数只有$v$,所以可以按照可分离变量来处理。

\section{一阶线性微分方程}

\subsection{线性方程}

形如$\dfrac{\textrm{d}y}{\textrm{d}x}+P(x)y=Q(x)$就是一阶线性方程。因为其对未知函数$y$与其导数都是一次方程。

若$Q(x)\equiv 0$,则是一阶齐次线性微分方程,可化为$\dfrac{\textrm{d}y}{y}=-P(x)\,\textrm{d}x$,$\ln y=\int P(x)\,\textrm{d}x+C'$,$y=e^{-\int P(x)\,\textrm{d}x}\cdot e^{C'}$,$y=Ce^{-\int P(x)\,\textrm{d}x}$。

若$Q(x)\neq 0$,则是一阶非齐次线性微分方程,令$y=C(x)e^{-\int P(x)\,\textrm{d}x}$,$C(x)$为关于$x$的具体函数,这是\textbf{常数变易法}。

代入$\dfrac{\textrm{d}y}{\textrm{d}x}+P(x)y=Q(x)$,得到$C(x)'e^{-\int P(x)\,\textrm{d}x}-C(x)e^{-\int P(x)\,\textrm{d}x}P(x)+P(x)C(x)e^{-\int P(x)\,\textrm{d}x}=Q(x)$,$C(x)'e^{-\int P(x)\,\textrm{d}x}=Q(x)$,从而得到$C(x)'$,再对$C(x)'$积分得到$C(x)=\displaystyle{\int Q(x)e^{\int P(x)\,\textrm{d}x}\,\textrm{d}x}+C$。从而代入$y=C(x)e^{-\int P(x)\,\textrm{d}x}$,得到\textcolor{aqua}{\textbf{定理:}}$y=e^{-\int P(x)\,\textrm{d}x}(\int Q(x)e^{\int P(x)\,\textrm{d}x}\,\textrm{d}x+C)$。非齐次通解就是其齐次通解加上一个非齐次的特解。

\textbf{例题:}求$\dfrac{\textrm{d}y}{\textrm{d}x}=\dfrac{1}{x+y}$。

解:不能直接做,因为不能分离出$y$。

可以两边求倒数:$\dfrac{\textrm{d}x}{\textrm{d}y}-x=y$,颠倒$xy$,得到$\dfrac{\textrm{d}y}{\textrm{d}x}-y=x$。就可以按照公式来求。

或令$x+y=u$,所以$y=u-x$,$\dfrac{\textrm{d}y}{\textrm{d}x}=\dfrac{\textrm{d}u}{\textrm{d}x}-1$,$\dfrac{\textrm{d}u}{\textrm{d}x}=\dfrac{1+u}{u}$,$\dfrac{u}{1+u}\textrm{d}u=\textrm{d}x$。

\textcolor{aqua}{\textbf{定理:}}一阶线性方程时,$\displaystyle{\int\dfrac{1}{x}\textrm{d}x}=\ln\vert x\vert=\ln x$。

证明:令$p=\dfrac{1}{x}$,$\int p\,\textrm{d}x=\displaystyle{\int\dfrac{1}{x}\textrm{d}x}=\ln\vert x\vert$。

根据公式$y=e^{-\ln\vert x\vert}(\int e^{\ln\vert x\vert}Q(x)\,\textrm{d}x+C)=\dfrac{1}{\vert x\vert}(\int\vert x\vert Q(x)\,\textrm{d}x+C)=\dfrac{1}{\pm x}$\\$(\int(\pm x)Q(x)\,\textrm{d}x+C)=\dfrac{1}{x}(\int xQ(x)\,\textrm{d}x\pm C)=\dfrac{1}{x}(\int xQ(x)\,\textrm{d}x+D)$。

同理$\ln y=\ln f(x)$,则$y=Cf(x)$。

\subsection{伯努利方程}

形如$\dfrac{\textrm{d}y}{\textrm{d}x}+P(x)y=Q(x)y^n$就是伯努利方程。若$y=1$则是可分离变量方程,若$y=0$则是一阶线性方程。

变形:$y^{-n}\dfrac{\textrm{d}y}{\textrm{d}x}+P(x)y^{1-n}=Q(x)$,又令$y^{1-n}=z$,$\dfrac{\textrm{d}z}{\textrm{d}x}=(1-n)y^{-n}\dfrac{\textrm{d}y}{\textrm{d}x}$,从而$\dfrac{1}{1-n}\dfrac{\textrm{d}z}{\textrm{d}x}=y^{-n}\dfrac{\textrm{d}y}{\textrm{d}x}$,代入$\dfrac{\textrm{d}y}{\textrm{d}x}+P(x)y=Q(x)y^n$得到$\dfrac{1}{1-n}\dfrac{\textrm{d}z}{\textrm{d}x}+P(x)z=Q(x)$,从而$\dfrac{\textrm{d}z}{\textrm{d}x}=(1-n)P(x)z=(1-n)Q(x)$,将$(1-n)P(x)$当作$P(x)$,$(1-n)Q(x)$当中$Q(x)$代入得到$z$的关系式,再利用上面线性方程的公式求$y$。

\section{可降阶的高阶微分方程}

高阶微分方程即含二阶以及二阶以上的微分方程,需要将其降为一阶微分方程。

\subsection{\texorpdfstring{$y^{(n)}=f(x)$}\ 型}

右边是只包含$x$的函数。

直接对函数不断求积分就可以了。连续积分$n$次,会得到一个含有$n$个任意常数的通解。这种方程没有特定出题考的意义。

\textbf{例题:}求$y'''=e^{2x}-\cos x$。

解:$y''=\dfrac{1}{2}e^{2x}-\sin x+C_1$,$y'=\dfrac{1}{4}e^{2x}+\cos x+C_1x+C_2$,$y=\dfrac{1}{8}e^{2x}+\sin x+\dfrac{1}{2}C_1x^2+C_2x+C_3$。

\subsection{\texorpdfstring{$y''=f(x,y')$}\ 型}

即存在$y''$,$y'$和$x$但是没有$y$。

所以令$y'=p$,$y''=p'$,代入:$p'=f(x,p)$,代入$p=\varphi(x,C_1)$,所以$\dfrac{\textrm{d}y}{\textrm{d}x}=\varphi(x,C_1)$,对其积分:$y=\int\varphi(x,C_1)\,\textrm{d}x+C_2$。

\textbf{例题:}求$(1+x^2)y''=2xy'$,满足初值条件$y\vert_{x=0}=1$,$y'\vert_{x=0}=3$的特解。

解:令$y'=p$,$y''=p'$,所以$(1+x^2)p'=2xp$。

$\dfrac{\textrm{d}p}{p}=\dfrac{2x}{1+x^2}\textrm{d}x$,$\ln p=\ln(1+x^2)+C'$,$p=C(1+x^2)$,所以$y'=3(1+x^2)$,$y=x^3+3x+1$。

\subsection{\texorpdfstring{$y''=f(y,y')$}\ 型}

即存在$y''$,$y'$和$y$但是没有$x$。

所以令$y'=p$,$y''=p'=\dfrac{\textrm{d}p}{\textrm{d}x}=\dfrac{\textrm{d}p}{\textrm{d}y}\cdot\dfrac{\textrm{d}y}{\textrm{d}x}=p\dfrac{\textrm{d}p}{\textrm{d}y}=f(y,p)$。

设其通解为$y'=p=\varphi(y,C_1)$。

分离变量并积分,得到通解为$\displaystyle{\int\dfrac{\textrm{d}y}{\varphi(y,C_1)}=x+C_2}$。

\textbf{例题:}求微分方程$yy''-y'^2=0$的通解。

解:令$y'=p$,$y''=p\dfrac{\textrm{d}p}{\textrm{d}y}$,代入$yp\dfrac{\textrm{d}p}{\textrm{d}y}-p^2=0$。

若$p\neq0$,$y\neq0$,则$yp\dfrac{\textrm{d}p}{\textrm{d}y}$,$\dfrac{\textrm{d}p}{p}=\dfrac{\textrm{d}y}{y}$,$p=Cy$。

若$p=0$,则$y'=0$,则$y$是一个常数。

所以综上$y=C_2e^{C_1y}$。

\section{高阶线性微分方程}

第一部分是一阶微分方程,分为可分离变量微分方程、齐次微分方程、一阶齐次线性微分方程、一阶非齐次线性微分方程。

第二部分是可降阶的高阶微分方程,是残缺项的高阶微分方程,分为三种。

第三部分就是本节的高阶线性微分方程,$y^{(n)}+a_1(x)y^{(n-1)}+\cdots+a_{n-1}(x)y'+a_n(x)y=0$就是$n$阶齐次线性微分方程,$y^{(n)}+a_1(x)y^{(n-1)}+\cdots+a_{n-1}(x)y'+a_n(x)y=f(x)$就是$n$阶非齐次线性微分方程

\subsection{概念}

\textcolor{violet}{\textbf{定义:}}方程$y''+P(x)y'+Q(x)y=f(x)$称为\textbf{二阶变系数线性微分方程},其中$P(x)$,$Q(x)$为系数函数,$f(x)$为自由项,都是已知的连续方程。

当$f(x)\equiv0$时,$y''+P(x)y'+Q(x)y=0$为\textbf{齐次方程}。

当$f(x)$不恒为0时,$y''+P(x)y'+Q(x)y=f(x)$为\textbf{非齐次方程}。

\textcolor{violet}{\textbf{定义:}}方程$y''+py'+qy=f(x)$称为\textbf{二阶常系数线性微分方程},其中$p$,$q$为常数,$f(x)$为自由项,都是已知的连续方程。

当$f(x)\equiv0$时,$y''+py'+qy=0$为\textbf{齐次方程}。

当$f(x)$不恒为0时,$y''+py'+qy=f(x)$为\textbf{非齐次方程}。

考试基本上只考常系数线性微分方程。

\subsection{解的结构}

若$\varphi_1(x)$与$\varphi_2(x)$为两个函数,当$\varphi_1(x)$与$\varphi_2(x)$不成比例,则称$\varphi_1(x)$与$\varphi_2(x)$线性无关,否则$\varphi_1(x)$与$\varphi_2(x)$线性相关。

\textcolor{aqua}{\textbf{定理:}}若$\varphi_1(x)$与$\varphi_2(x)$为$y^{(n)}+a_1(x)y^{(n-1)}+\cdots+a_{n-1}(x)y'+a_n(x)y=0$的解,则$y=C_1\varphi_1(x)+C_2\varphi_2(x)$也为其解。

证明:因为$\varphi_1(x)$与$\varphi_2(x)$为解,所以代入方程:

$\varphi_1''+a(x)\varphi_1'+b(x)\varphi_1=0$,$\varphi_2''+a(x)\varphi_2'+b(x)\varphi_2=0$

从而$(C_1\varphi_1+C_2\varphi_2)''+a(x)(C_1\varphi_1+C_2\varphi_2)'+b(x)(C_1\varphi_1+C_2\varphi_2)=C_1(\varphi_1''+a(x)\varphi_1'+b(x)\varphi_1)+C_2(\varphi_2''+a(x)\varphi_2'+b(x)\varphi_2)=0$。

所以得证。

\textcolor{aqua}{\textbf{定理:}}若$\varphi_1(x)$与$\varphi_2(x)$分别为$y^{(n)}+a_1(x)y^{(n-1)}+\cdots+a_{n-1}(x)y'+a_n(x)y=0$与$y^{(n)}+a_1(x)y^{(n-1)}+\cdots+a_{n-1}(x)y'+a_n(x)y=f(x)$的解,则$y=\varphi_1(x)+\varphi_2(x)$为$y^{(n)}+a_1(x)y^{(n-1)}+\cdots+a_{n-1}(x)y'+a_n(x)y=f(x)$的解。

证明:$\varphi_1''+a(x)\varphi_1'+b(x)\varphi_1=0$,$\varphi_2''+a(x)\varphi_2'+b(x)\varphi_2=f(x)$,代入$y=\varphi_1(x)+\varphi_2(x)$:

$(\varphi_1+\varphi_2)''+a(x)(\varphi_1+\varphi_2)'+b(x)(\varphi_1+\varphi_2)=(\varphi_1''+a(x)\varphi_1'+b(x)\varphi_1)+(\varphi_2''+a(x)\varphi_2'+b(x)\varphi_2)=f(x)$。

所以得证。

\textcolor{aqua}{\textbf{定理:}}若$\varphi_1(x)$与$\varphi_2(x)$为$y^{(n)}+a_1(x)y^{(n-1)}+\cdots+a_{n-1}(x)y'+a_n(x)y=f(x)$的解,则$y=\varphi_1(x)-\varphi_2(x)$为$y^{(n)}+a_1(x)y^{(n-1)}+\cdots+a_{n-1}(x)y'+a_n(x)y=0$的解。

证明:$\varphi_1''+a(x)\varphi_1'+b(x)\varphi_1=f(x)$,$\varphi_2''+a(x)\varphi_2'+b(x)\varphi_2=f(x)$,代入$y=\varphi_1(x)-\varphi_2(x)$:

$(\varphi_2-\varphi_1)''+a(x)(\varphi_2-\varphi_1)'+b(x)(\varphi_2-\varphi_1)$

$(\varphi_2''+a(x)\varphi_2'+b(x)\varphi_2)-(\varphi_1''+a(x)\varphi_1'+b(x)\varphi_1)$

$f(x)-f(x)=0$,所以得证。

\textcolor{aqua}{\textbf{定理:}}若$\varphi_1(x)$与$\varphi_2(x)$分别为$y^{(n)}+a_1(x)y^{(n-1)}+\cdots+a_{n-1}(x)y'+a_n(x)y=f_1(x)$与$y^{(n)}+a_1(x)y^{(n-1)}+\cdots+a_{n-1}(x)y'+a_n(x)y=f_2(x)$的解,则$y=\varphi_1(x)+\varphi_2(x)$为$y^{(n)}+a_1(x)y^{(n-1)}+\cdots+a_{n-1}(x)y'+a_n(x)y=f_1(x)+f_2(x)$的解。

\subsection{二阶常系数齐次线性微分方程的通解}

可以根据高阶微分方程的解的结构得到二阶的通解。

对于$y''+py'+qy=0$,其对应的特征方程为$\lambda^2+p\lambda+q=0$,求其特征根,有三种情况($\lambda_1\lambda_2$为任意常数):

\begin{enumerate}
    \item 若$p^2-4q>0$,设$\lambda_1,\lambda_2$是特征方程的两个不等实根,即$\lambda_1\neq\lambda_2$,其通解为$y=C_1e^{\lambda_1x}+C_2e^{\lambda_2x}$。
    \item 若$p^2-4q=0$,设$\lambda_1,\lambda_2$是特征方程的两个相等实根,即二重根,令$\lambda=\lambda_1=\lambda_2$,其通解为$y=(C_1+C_2x)e^{\lambda x}$。
    \item 若$p^2-4q<0$,设$\alpha\pm\beta i$是特征方程的一对共轭复根,$\lambda_{1,2}=\dfrac{-p\pm\sqrt{4q-p^2}i}{2}$\\$=-\dfrac{p}{2}\pm\dfrac{\sqrt{4q-p^2}}{2}i$,记为$\alpha\pm\beta i$,其通解为$y=e^{\alpha x}(C_1\cos\beta x+C_2\sin\beta x)$。
\end{enumerate}

\subsection{二阶常系数非齐次线性微分方程的特解}

二阶常系数非齐次线性微分方程的通解就是对应齐次方程的通解加上非齐次方程的特解

设$P_n(x)$,$P_m(x)$分别为$x$的$n$次、$m$次多项式。

\begin{enumerate}
    \item 当自由项$f(x)=P_n(x)e^{\alpha x}$时,特解设为$y^*=e^{\alpha x}Q_n(x)x^k$,其中$e^{\alpha x}$照抄,$Q_n(x)$为$x$的$n$次多项式,且$k=\left\{\begin{array}{ll}
        0, & \alpha\text{不是特征根} \\
        1, & \alpha\text{是单特征根} \\
        2, & \alpha\text{是二重特征根}
    \end{array}\right.$。
    \item 当自由项$f(x)=e^{\alpha x}[P_m(x)\cos\beta x+P_n(x)\sin\beta x]$时,特解设为$y^*=e^{\alpha x}[Q_l^{(1)}(x)\cos\beta x+Q_l^{(2)}(x)\sin\beta x]x^k$,其中$e^{\alpha x}$照抄,$l=\max\{m,n\}$,$Q_l^{(1)}$、$Q_l^{(2)}$为$x$的两个不同的$l$次多项式,且$k=\left\{\begin{array}{ll}
        0, & \alpha\pm\beta i\text{不是特征根} \\
        1, & \alpha\pm\beta i\text{是特征根} \\
    \end{array}\right.$。
\end{enumerate}

最后求导代回原式得到系数值。

对于第二种需要举例说明一下,如果$f(x)=e^{7x}[(x^2+3x)\cos4x+(x+6)\sin4x]$,其特解设法:首先$e^{7x}$直接写过来;然后判断多项式系数,第一个多项式最高次数为2,第二个多项式最高次数为1,所以$l$为较高那个即2,所以设为$e^{7x}[(A_1x^2+B_1x+C)\cos4x+(A_2x^2+B_2x+C)\sin4x]$(当然一般不会这么高,基本上都是一次的$A\cos\beta x+B\sin\beta x$的形式);最后判断其特征根,$7\pm4i$是否为特征根,如果$\Delta>0$特征根$\lambda$是实数就肯定不是,不需要算直接$k=0$。

\section{欧拉方程}

\subsection{概念}

\textcolor{violet}{\textbf{定义:}}形如$x^2y''+pxy'+qy=f(x)$的方程称为\textbf{欧拉方程},其中$pq$为常数,$f(x)$为已知函数。

注意与一般的二阶常系数非齐次线性微分方程$y''+py'+qy=f(x)$对比,其二阶导和一阶导前面多了$x$的多项式。

% $x^2\dfrac{\textrm{d}^2y}{\textrm{d}x^2}+px\dfrac{\textrm{d}y}{\textrm{d}x}+qy=f(x)$

\subsection{解法}

使用换元法$x=e^t$将欧拉方程换为二阶常系数非齐次线性方程。

当$x>0$时,令$x=e^t$,则$t=\ln x$,$\dfrac{\textrm{d}t}{\textrm{d}x}=\dfrac{1}{x}$,$\dfrac{\textrm{d}y}{\textrm{d}x}=\dfrac{\textrm{d}y}{\textrm{d}t}\dfrac{\textrm{d}t}{\textrm{d}x}=\dfrac{1}{x}\dfrac{\textrm{d}y}{\textrm{d}t}$,$\dfrac{\textrm{d}^2y}{\textrm{d}x^2}=\dfrac{\textrm{d}}{\textrm{d}x}\left(\dfrac{1}{x}\dfrac{\textrm{d}y}{\textrm{d}t}\right)=-\dfrac{1}{x^2}\dfrac{\textrm{d}y}{\textrm{d}t}+\dfrac{1}{x}\dfrac{\textrm{d}}{\textrm{d}x}\left(\dfrac{\textrm{d}y}{\textrm{d}t}\right)=-\dfrac{1}{x^2}\dfrac{\textrm{d}y}{\textrm{d}t}+\dfrac{1}{x^2}\dfrac{\textrm{d}^2y}{\textrm{d}t^2}$,方程化为$\dfrac{\textrm{d}^2y}{\textrm{d}t^2}+(p-1)\dfrac{\textrm{d}y}{\textrm{d}t}+qy=f(e^t)$,解出结果,组后用$t=\ln x$回代。

当$x<0$是,令$x=-e^t$,同理可得。

\textbf{例题:}求欧拉方程$x^2\dfrac{\textrm{d}^2y}{\textrm{d}x^2}+4x\dfrac{\textrm{d}y}{\textrm{d}x}+2y=0$($x>0$)的通解。

解:可以直接利用公式,变为$\dfrac{\textrm{d}^2y}{\textrm{d}t^2}+3\dfrac{\textrm{d}y}{\textrm{d}t}+2y=0$

即$y''+3y'+2y=0$,特征方程$\lambda^2+3\lambda+2=0$,$\lambda_1=-1$,$\lambda_2=-2$。

$\therefore y=C_1e^{-x}+C_2e^{-2x}$。代入$x=e^t$,$y=\dfrac{C_1}{x}+\dfrac{C_2}{x^2}$。

%\end{document}
