%空头
% UTF8编码,ctexart现实中文
%删除了\usepackage{color}
% 使用颜色
\definecolor{orange}{RGB}{255,127,0} 
\definecolor{violet}{RGB}{192,0,255} 
\definecolor{aqua}{RGB}{0,255,255} 
%删除了\usepackage{geometry}
\setcounter{tocdepth}{4}
\setcounter{secnumdepth}{4}
% 设置四级目录与标题
%删除A4
% 默认大小为A4
%删除页面边距
% 默认页边距为1英尺与1.25英尺
%删除了\usepackage{indentfirst}
%删除首行缩进
% 首行缩进2个中文字符
%删除\usepackage{setspace}
\renewcommand{\baselinestretch}{1.5}
% 1.5倍行距
%删除\usepackage{amssymb}
% 因为所以
%删除\usepackage{amsmath}
% 数学公式
%\usepackage[colorlinks,linkcolor=black,urlcolor=blue]{hyperref}
% 超链接
%删除了作者
\chapter{向量代数与空间解析几何}
%删除了页面格式
% \section{向量代数}

\section{空间解析几何}

\subsection{平面方程}

\subsection{直线方程}

\subsection{位置关系}

\subsection{空间曲线}

\subsubsection{投影}

\subsection{空间曲面}

\section{场论初步}

\subsection{方向导数}

\subsection{梯度}

\subsection{散度与旋度}

直接代入公式。

\textbf{例题:}计算向量场$u(x,y,z)=xy^2i+ye^xj+x\ln(1+z^2)k$在点$P(1,1,0)$的散度和旋度。

解:所以$u(x,y,z)=(P,Q,R)$,$P=xy^2$,$Q=ye^x$,$R=x\ln(1+z^2)$。

$\dfrac{\partial P}{\partial x}=y^2$,$\dfrac{\partial Q}{\partial y}=e^x$,$\dfrac{\partial R}{\partial z}=\dfrac{2zx}{1+z^2}$。

代入$P(1,1,0)$,得到散度$\textrm{div}\,\vec{u}=1+e$。

旋度$\overrightarrow{\textrm{rot}}\,\vec{u}=\left\vert\begin{array}{ccc}
    \vec{i} & \vec{j} & \vec{k} \\
    \dfrac{\partial}{\partial x} & \dfrac{\partial}{\partial y} & \dfrac{\partial}{\partial z} \\
    xy^2 & ye^x & x\ln(1+z^2)
\end{array}\right\vert=\dfrac{\partial x\ln(1+z^2)}{\partial y}\vec{i}+\dfrac{\partial xy^2}{\partial z}\vec{j}+\dfrac{\partial ye^x}{\partial x}\vec{k}-\dfrac{\partial xy^2}{\partial y}\vec{k}-\dfrac{\partial ye^x}{\partial z}\vec{i}-\dfrac{\partial x\ln(1+z^2)}{\partial x}\vec{j}=0+0+ye^x\vec{k}-2xy\vec{k}-0-\ln(1+z^2)\vec{j}=-\ln(1+z^2)\vec{j}+(ye^x-2xy)\vec{k}=(0,0,e-2)$。

%\end{document}
