%空头
% UTF8编码,ctexart现实中文
%删除了\usepackage{color}
% 使用颜色
%删除了\usepackage{geometry}
\setcounter{tocdepth}{4}
\setcounter{secnumdepth}{4}
% 设置四级目录与标题
%删除A4
% 默认大小为A4
%删除页面边距
% 默认页边距为1英尺与1.25英尺
%删除了\usepackage{indentfirst}
%删除首行缩进
% 首行缩进2个中文字符
%删除\usepackage{setspace}
\renewcommand{\baselinestretch}{1.5}
% 1.5倍行距
%删除\usepackage{amssymb}
% 因为所以
%删除\usepackage{amsmath}
% 数学公式
%\usepackage[colorlinks,linkcolor=black,urlcolor=blue]{hyperref}
% 超链接
%删除了作者
\chapter{数理统计}
%删除了页面格式
\section{统计量}

利用期望和方差等数学特征之间的关系进行计算统计量,往往以$\sum\limits_{i=1}^nX_i$或类似的形式。

\textbf{例题:}已知总体$X$的期望为$EX=0$,方差$DX=\sigma^2$。从总体抽取容量为$n$的简单随机样本,其均值和方差分别为$\overline{X}$,$S^2$。记$S_k^2=\dfrac{n}{k}\overline{X}^2+\dfrac{1}{k}S^2$($k=1,2,3,4$),则()。

$A.E(S_1^2)=\sigma^2$\qquad$B.E(S_2^2)=\sigma^2$

$C.E(S_3^2)=\sigma^2$\qquad$D.E(S_4^2)=\sigma^2$

解:$E(S_k^2)=E\left(\dfrac{n}{k}\overline{X}^2+\dfrac{1}{k}S^2\right)=\dfrac{n}{k}E\overline{X}^2+\dfrac{1}{k}E(S^2)=\dfrac{n}{k}((E\overline{X})^2+D\overline{X})+\dfrac{1}{k}E(S^2)=\dfrac{n}{k}\left(0+\dfrac{\sigma^2}{n}\right)+\dfrac{1}{k}\sigma^2=\dfrac{2\sigma^2}{k}$,$\therefore k=2$。

\textbf{例题:}设$X_i$为来自总体$E(\lambda)$($\lambda>0$)的简单随机样本,记统计量$T=\dfrac{1}{n}\sum\limits_{i=1}^nX_i^2$,求$ET$。

解:$ET=E\dfrac{1}{n}\sum\limits_{i=1}^nX_i^2=\dfrac{1}{n}\sum\limits_{i=1}^nEX_i^2=\dfrac{1}{n}\sum\limits_{i=1}^n(DX_i+E^2X_i)=\dfrac{1}{n}\sum\limits_{i=1}^n\left(\dfrac{1}{\lambda^2}+\dfrac{1}{\lambda^2}\right)\\=\dfrac{1}{n}\cdot\dfrac{2n}{\lambda^2}=\dfrac{2}{\lambda^2}$。

\textbf{例题:}设$X_i$为来自总体$X$的简单随机样本,而$X\sim B\left(1,\dfrac{1}{2}\right)$。记$\overline{X}=\dfrac{1}{n}\sum\limits_{i=1}^nX_i$,求$P\left\{\overline{X}=\dfrac{k}{n}\right\}$。($0\leqslant k\leqslant n$)

解:$\because X\sim B\left(1,\dfrac{1}{2}\right)$,$\therefore\sum\limits_{i=1}^nX_i\sim B\left(n,\dfrac{1}{2}\right)$。

$P\left\{\overline{X}=\dfrac{k}{n}\right\}=P\left\{\dfrac{1}{n}\sum\limits_{i=1}^nX_i=\dfrac{k}{n}\right\}=P\left\{\sum\limits_{i=1}^nX_i=k\right\}=C_n^k\left(\dfrac{1}{2}\right)^k\left(\dfrac{1}{2}\right)^{n-k}\\=C_n^k\cdot\left(\dfrac{1}{2}\right)^n$。

\section{三大分布}

\subsection{\texorpdfstring{$\chi^2$分布}{}}

\textbf{例题:}设$X_1,X_2,X_3,X_4$是来自正态总体$N(0,4)$的简单随机样本,记$X=a(X_1-2X_2)^2+b(3X_3-4X_4)^2$。求$X$服从$\chi^2$分布下的参数与自由度。

解:若$X_1,X_2,X_3,X_4$同一个正态分布,所以$EX_1=EX_2=EX_3=EX_4=0$,$DX_1=DX_2=DX_3=DX_4=4$。

$E(X_1-2X_2)=EX_1-2EX_2=0$,$D(X_1-2X_2)=DX_1-4DX_2=20$。

$\therefore X_1-2X_2\sim N(0,20)$,同理$3X_3-4X_4\sim N(0,100)$。

对其标准化:$\dfrac{X_1-2X_2-0}{\sqrt{20}}\sim N(0,1)$,$\dfrac{3X_3-4X_4-0}{\sqrt{100}}\sim N(0,1)$。

若要让$X$满足$\chi^2$分布,则要将$a(X_1-2X_2)^2+b(3X_3-4X_4)^2$两项标准化。

$\therefore\dfrac{(X_1-2X_2)^2}{20}+\dfrac{(3X_3-4X_4)^2}{100}\sim\chi^2(2)$,所以$a=\dfrac{1}{20}$,$b=\dfrac{1}{100}$。

\subsection{\texorpdfstring{$t$分布}{}}

\textbf{例题:}设$X_1,X_2,\cdots,X_8$是来自正态总体$N(0,3^2)$的简单随机样本,则统计量$Y=\dfrac{X_1+X_2+X_3+X_4}{\sqrt{X_5^2+X_6^2+X_7^2+X_8^2}}$服从什么分布?

解:$\because X_1,\cdots,X_8\sim N(0,9)$,$\therefore X_1+X_2+X_3+X_4\sim N(0,36)$。

$\therefore\dfrac{X_1+X_2+X_3+X_4-0}{6}\sim N(0,1)$。

$\dfrac{X_5^2+X_6^2+X_7^2+X_8^2}{9}=\left(\dfrac{X_5-0}{3}\right)^2+\left(\dfrac{X_6-0}{3}\right)^2+\left(\dfrac{X_7-0}{3}\right)^2+\left(\dfrac{X_8-0}{3}\right)^2$\\$\sim\chi^2(4)$

$\therefore\dfrac{\dfrac{X_1+X_2+X_3+X_4-0}{6}}{\sqrt{\dfrac{X_5^2+X_6^2+X_7^2+X_8^2}{9}/4}}=\dfrac{X_1+X_2+X_3+X_4}{\sqrt{X_5^2+X_6^2+X_7^2+X_8^2}}\sim t(4)$。

\subsection{\texorpdfstring{$F$分布}{}}

\textbf{例题:}设$X_1,X_2,\cdots,X_15$是来自正态总体$N(0,3^2)$的简单随机样本,则统计量$Y=\dfrac{X_1^2+X_2^2+\cdots+X_{10}^2}{2X_{11}^2+X_{12}^2+\cdots+X_{15}^2}$服从什么分布?

解:$\because\dfrac{X_i-0}{3}\sim N(0,1)$,$\left(\dfrac{X_i-0}{3}\right)^2=\dfrac{x_i^2}{9}\sim\chi^2(1)$。

$\therefore\dfrac{X_1^2+X_2^2+\cdots+X_{10}^2}{9}\sim\chi^2(10)$,$\dfrac{X_{11}^2+X_{12}^2+\cdots+X_{15}^2}{9}\sim\chi^2(5)$。

$\therefore\dfrac{\dfrac{X_1^2+X_2^2+\cdots+X_{10}^2}{9}/10}{\dfrac{X_{11}^2+X_{12}^2+\cdots+X_{15}^2}{9}/5}=\dfrac{X_1^2+X_2^2+\cdots+X_{10}^2}{2X_{11}^2+X_{12}^2+\cdots+X_{15}^2}=Y\sim F(10,5)$。

\textbf{例题:}已知$(X,Y)$的概率分布函数为$f(x,y)=\dfrac{1}{2\pi}e^{-\frac{1}{2}(x^2+y^2-2y+1)}$,$x,y\in R$,求$\dfrac{X^2}{(Y-1)^2}$的分布。

解:$f(x,y)=\dfrac{1}{2\pi}e^{-\frac{1}{2}(x^2+y^2-2y+1)}=\dfrac{1}{2\pi}e^{-\frac{1}{2}(x^2+(y-1)^2)}$,所以根据二维正态分布的形式,得到$(X,Y)\sim(0,1;1,1;0)$。

即$X\sim\varPhi(x)$,$Y-1\sim\varPhi(x)$,$\therefore X^2\sim\chi^2(1)$,$(Y-1)^2\sim\chi^2(1)$,$\therefore\dfrac{X^2}{(Y-1)^2}\sim F(1,1)$。

\subsection{函数分布}

\textbf{例题:}设随机变量$X\sim t(n)$,$Y\sim F(1,n)$,常数$C$使得$P\{X>C\}=0.6$,求$P\{Y>C^2\}$。

解:$X\sim t(n)$,则$X=\dfrac{X_1}{\sqrt{Y_1/n}}\sim t(n)$,其中$X_1\sim N(0,1)$,$Y_1\sim\chi^2(n)$。

$\therefore X^2=\dfrac{X_1^2}{Y_1/n}=\dfrac{X_1^2/1}{Y_1/n}\sim\dfrac{\chi^2(1)/1}{\chi^2(n)/n}=F(1,n)$。

又$P\{Y>C^2\}=1-P\{Y\leqslant C^2\}$。$P\{X^2>C^2\}=1-P\{X^2\leqslant C^2\}$。

又$P\{X^2\leqslant C^2\}=P\{-C\leqslant X\leqslant C\}$,根据偶函数性质$=0.2$。

$\therefore P\{X^2>C^2\}=0.8$。

\section{参数估计}

\subsection{矩估计}

基本方法就是$EX=\dfrac{1}{n}\sum\limits_{i=1}^nX_i$。

如果只有一个参数就使用一阶矩,如果有两个参数就使用二阶矩,一般不会超过两个未知数。

即$EX=\dfrac{1}{n}\sum\limits_{i=1}^nX_i=\overline{X}=\hat{\mu}$,$EX^2=\dfrac{1}{n}\sum\limits_{i=1}^nX_i^2=\hat{\sigma}^2+\hat{\mu}^2$。

$\hat{\sigma}^2=EX^2-(EX)^2=\dfrac{1}{n}\sum\limits_{i=1}^nX_i^2-\overline{X}^2=\dfrac{1}{n}\sum\limits_{i=1}^n(X_i-\overline{X})^2$。

\subsubsection{一阶矩}

\subsubsection{二阶矩}

\textbf{例题:}设$X_i$为来自区间$[-a,a]$上均匀分布的总体$X$的简单随机样本,求$a$的矩估计量。

解:首先矩估计就是$E(X^k)=\dfrac{1}{n}\sum\limits_{i=1}^nX_i^k$。

又对于均匀分布$X_i\sim U(-a,a)$,$EX=\dfrac{a+b}{2}=0$,$DX=\dfrac{(b-a)^2}{12}=\dfrac{a^2}{3}$。

所以$EX$不含有$a$,使用二阶矩$EX^2=DX+E^2X=\dfrac{a^2}{3}=\dfrac{1}{n}\sum\limits_{i=1}^nX_i^2$。

解得$a=\sqrt{\dfrac{3}{n}\sum\limits_{i=1}^nX_i^2}$。

\subsection{最大似然估计}

步骤:写出概率函数或密度函数;写出似然函数(代入观测值$x_i$并连乘);两边取对数;求导数并令为0求出表达式。

\subsubsection{离散型}

\textbf{例题:}设总体$X$的概率分布为:\medskip

\begin{tabular}{c|cccc}
    \hline
    $X$ & 0 & 1 & 2 & 3 \\ \hline
    $P$ & $\theta^2$ & $2\theta(1-\theta)$ & $\theta^2$ & $1-2\theta$ \\ \hline
\end{tabular} \medskip

其中$\theta\int\left(0,\dfrac{1}{2}\right)$为未知参数,从总体$X$中抽取容量为8的一组样本,其样本值为3,1,3,0,3,1,2,3。求$\theta$的最大似然估计值。

解:

根据样本值,可以得出:\medskip

\begin{tabular}{c|cccc}
    \hline
    $X$ & 0 & 1 & 2 & 3 \\ \hline
    次数 & 1 & 2 & 1 & 4 \\ \hline
\end{tabular} \medskip

将所有的概率相乘:$L(\theta)l=(1-2\theta)^4[2\theta(1-\theta)]^2\cdot\theta^2\cdot\theta^2=4\theta^6(1-\theta)^2(1-2\theta)^4$。

对其求对数:$\ln L(\theta)=\ln4+6\ln\theta+2\ln(1-\theta)+4\ln(1-2\theta)$。

对其求导:$\dfrac{\textrm{d}\ln L(\theta)}{\textrm{d}\theta}=\dfrac{6}{\theta}-\dfrac{2}{1-\theta}-\dfrac{8}{1-2\theta}=0$。解得$\theta=\dfrac{7\pm\sqrt{13}}{12}$。

$0<\theta<\dfrac{1}{2}$,舍去正值,得到$\hat{\theta}=\dfrac{7-\sqrt{13}}{12}$。

\subsubsection{连续型}

\textbf{例题:}设随机变量$X$在区间$[0,\theta]$上服从均匀分布,$X_1,X_2,\cdots,X_n$是来自$X$的简单随机样本,求$\theta$的最大似然估计量$\hat{\theta}$

解:$X\sim U(0,\theta)$,$f(x)=\left\{\begin{array}{ll}
    \dfrac{1}{\theta}, & 0<x<\theta \\
    0, & \text{其他}
\end{array}\right.$,$L(\theta)=\left\{\begin{array}{ll}
    \dfrac{1}{\theta^n}, & 0<x_i<\theta \\
    0, & \text{其他}
\end{array}\right.$。

求$\hat{\theta}$即求$L(\theta)$的最大值,$\theta$的最小值。又必然$0<x_i<\theta$。

所以$\hat{\theta}=\max x_i$,即$\theta$的最大似然估计为$\max\limits_{1\leqslant i\leqslant n}X_i$。

(取最大值而不是最小值是因为为保证所有$x_i$都在定义域上,$0<x_i<\theta$,所以要求$\theta>\max x_i$)

\textbf{例题:}设$X_1,X_2,\cdots X_n$是来自总体$X$的简单随机样本,$X$的概率密度函数$f(x)=\dfrac{1}{2\lambda}e^{-\frac{\vert x\vert}{\lambda}}$,$x\in R$,$\lambda>0$,求$\lambda$的最大似然估计量$\hat{\lambda}$。

解:$\because f(x)=\dfrac{1}{2\lambda}e^{-\frac{\vert x\vert}{\lambda}}$,$\therefore L(\lambda)=\prod\limits_{i=1}^n\dfrac{1}{2\lambda}e^{-\frac{\vert x\vert}{\lambda}}=\left(\dfrac{1}{2\lambda}\right)^ne^{-\frac{1}{\lambda}\sum\limits_{i=1}^n\vert x_i\vert}$。

$\ln L(\lambda)=-n\ln2-n\ln\lambda-\dfrac{1}{\lambda}\sum\limits_{i=1}^n\vert x_i\vert$,$\dfrac{\textrm{d}\ln L(\lambda)}{\textrm{d}\lambda}=-\dfrac{n}{\lambda}+\dfrac{1}{\lambda^2}\sum\limits_{i=1}^n\vert x_i\vert$。

令$\dfrac{\textrm{d}\ln L(\lambda)}{\textrm{d}\lambda}=0$,则$\dfrac{n}{\lambda}=\dfrac{1}{\lambda^2}\sum\limits_{i=1}^n\vert x_i\vert$,解得$\lambda=\dfrac{1}{n}\sum\limits_{i=1}^n\vert x_i\vert$。

即$\hat{\lambda}=\dfrac{1}{n}\sum\limits_{i=1}^n\vert X_i\vert$。

\section{估计量评价标准}

\subsection{无偏性}

$E\hat{\theta}=\theta$。

\subsection{有效性}

$D\hat{\theta_1}<D\hat{\theta_2}$。

\subsection{一致性}

\section{置信区间}

\subsection{方差已知}

\textbf{例题:}一批零件的长度服从正态分布$N(\mu,\sigma^2)$,其中$\mu,\sigma^2$均未知。现从中随机抽取16个零件,测得样本均值$\overline{x}=20cm$,样本标准差为$s=1cm$,求$\mu$的置信水平为0.90的置信区间。

解:$\sigma$未知,所以使用$s$来求置信空间。

置信空间为$(\overline{X}-t_\frac{\alpha}{2}(n-1)\dfrac{S}{\sqrt{n}},\overline{X}+t_\frac{\alpha}{2}(n-1)\dfrac{S}{\sqrt{n}})$。

已知$\overline{x}=20$,$s=1$,$n=16$,$\alpha=1-0.90=0.1$。

所以置信空间为$\left(20-\dfrac{1}{4}t_{0.05}(15),20+\dfrac{1}{4}t_{0.05}(15)\right)$。

\subsection{方差未知}

\textbf{例题:}设某群人的年龄$X\sim N(\mu,\sigma^2)$,随机了解到五个人的年龄:39,54,61,72,59,求均值$\mu$的置信度为$0.95$的置信区间。

解:由于$\sigma$未知,所以使用样本方差,$\dfrac{\overline{X}-\mu}{S/\sqrt{n}}\sim t(n-1)$。

其中置信区间为$\left(\overline{X}-\dfrac{S}{\sqrt{n}}t_{0.025}(n-1),\overline{X}+\dfrac{S}{\sqrt{n}}t_{0.025}(n-1)\right)$。

又$\overline{x}=\dfrac{1}{5}(39+54+61+72+59)=57$,$S=\sqrt{\dfrac{1}{n-1}\sum\limits_{i=1}^5(x_i-\overline{x})}=12$。

其中$t_{0.025}(n-1)=t_{0.025}(4)=2.7764$,所以代入得到$(42.13,71,87)$。

\section{假设检验}

\textbf{例题:}设考试成绩服从正态分布,随机抽取36位考生成绩,平均分为66.5分,标准差为15分。在显著性水平0.05下是否可以认为这次考试的平均水平为70分。

解:首先提出假设$H_0:\mu=70$,$H_1:\mu\neq70$。

将$X$使用样本标准差进行标准化:$T=\dfrac{\overline{X}-\mu}{S/\sqrt{n}}\sim t(n-1)$。

给定显著性水平$0.05$,写出拒绝域$T<-t_{\frac{\alpha}{2}}(n-1)$或$T>t_{\frac{\alpha}{2}}(n-1)$。

代入计算统计量,$\vert T\vert=\left\vert\dfrac{\overline{X}-\mu}{S/\sqrt{n}}\right\vert=\left\vert\dfrac{66.5-70}{15/6}\right\vert=1.4$。

又$t_{\frac{\alpha}{2}}(n-1)=t_{0.05}(35)=2.0301>1.4$不在拒绝域内,所以接受原假设。

即可以认为平均水平为70分。

\textbf{例题:}设$X_1,X_2,\cdots,X_{36}$是取自正态总体$N(\mu,0.04)$的简单随机样本,其中$\mu$为未知参数,即$\overline{X}=\dfrac{1}{36}\sum\limits_{i=1}^{36}X_i$,若对于检验问题$H_0:\mu\leqslant0.5$,$H_1:\mu>0.5$在显著性水平$\alpha=0.05$,取得检验拒绝域$D=\{(x_1,x_2,\cdots,x_{36}):\overline{x}>C\}$,求$C$。

解:当$H_0$成立,则$X\sim N(0.5,0.04)$,$\overline{X}\sim N(0.5,0.04\div36)=N\left(\dfrac{1}{2},\dfrac{1}{900}\right)$。

$\alpha=0.05=P\{$拒绝$H_0|H_0$成立$\}=P\{\overline{X}>C\}=1-P\{\overline{X}\leqslant C\}=1-\varPhi((C-0.5)\times30)=1-\varPhi(30C-15)$。

$\therefore\varPhi(30C-15)=0.95=\varPhi(1.645)$,即$30C-15=1.645$,$C=0.5548$。

\textbf{例题:}已知某机器生产出来的零件长度$X$(单位:$cm$)服从正态分布$N(\mu,\delta^2)$,现从中随意抽取容量为16的一个样本,测得样本均值$\overline{x}=10$,样本方差$s^2=0.16$,$t_{0.025}(15)=2.132$。

(1)求总体均值$\mu$置信水平为0.95的置信区间。

(2)在显著性水平$0.05$下检验假设$H_0:\mu=9.7$,$H_1:\mu\neq9.7$。

(1)解:根据公式直接解出置信空间$(10-0.1t_{0.025}(15),10+0.1t_{0.025}(15))=(9.7868,10.2132)$。

(2)解:根据假设$H_0$,得到拒绝域$(-\infty,9.4868]\cup[9.9132,+\infty)$。

又$\overline{X}=10$在拒绝域$[9.9132,+\infty)$上,所以假设$H_0$拒绝。

\section{两类错误}

\textbf{例题:}假定$X$是连续型随机变量,$U$是对$X$的一次观测值,关于其概率密度$f(x)$有如下假设:

$H_0:f(x)=\left\{\begin{array}{ll}
    \dfrac{1}{2}, & 0\leqslant x\leqslant2 \\
    0, & \text{其他}
\end{array}\right.$,$H_1:f(x)=\left\{\begin{array}{ll}
    \dfrac{x}{2}, & 0\leqslant x\leqslant2 \\
    0, & \text{其他}
\end{array}\right.$。

检验规则:当事件$V=\left\{U>\dfrac{3}{2}\right\}$出现时,否定假设$H_0$,接受$H_1$,求犯第一类错误概率和第二类错误概率$\alpha\beta$。

解:$\alpha=P\left\{U>\dfrac{3}{2}\bigg|H_0\right\}=\displaystyle{\int_\frac{3}{2}^2\dfrac{1}{2}\,\textrm{d}x=\dfrac{1}{4}}$。

$\beta=P\left\{U\leqslant\dfrac{3}{2}\bigg|H_1\right\}=\displaystyle{\int_0^{\frac{3}{2}}\dfrac{x}{2}\,\textrm{d}x=\dfrac{9}{16}}$。

%\end{document}
